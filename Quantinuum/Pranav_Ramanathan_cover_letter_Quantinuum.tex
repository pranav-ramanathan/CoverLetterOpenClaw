\documentclass[10pt,a4paper]{article}
\usepackage[margin=0.8in]{geometry}
\usepackage[T1]{fontenc}
\usepackage[utf8]{inputenc}
\usepackage{lmodern}
\usepackage{hyperref}
\usepackage{xcolor}
\usepackage{setspace}
\pagestyle{empty}
\hypersetup{colorlinks=true,urlcolor=blue}
\setstretch{0.98}
\setlength{\parskip}{0.45em}
\setlength{\parindent}{0pt}

\begin{document}

{\LARGE \textbf{Pranav Ramanathan}}\\
+44 7916 717 471\;|\;\href{mailto:pranav.ramanathan@gmail.com}{pranav.ramanathan@gmail.com}\\
\href{https://linkedin.com/in/pranav-ramanathan/}{LinkedIn}\;|\;
\href{https://github.com/pranav-ramanathan}{GitHub}\;|\;
\href{https://pranavramanathan.me}{Portfolio}

\vspace{0.4cm}
\today

\vspace{0.4cm}
Hiring Team\\
Quantinuum

\vspace{0.3cm}
Dear Hiring Team,

I am writing to express my interest in internship opportunities at Quantinuum for 2026 in the UK. I am completing a BSc in Mathematics at Queen Mary University of London (predicted First Class, graduating in 2026) and have accepted an MSc in Artificial Intelligence beginning in 2026. I am motivated by Quantinuum's integrated full-stack approach, combining high-fidelity hardware with advanced software to accelerate practical quantum computing.

My recent work has focused on high-performance ML systems, optimisation, and rigorous experimentation. During my STRIDE research internship at QMUL, I benchmarked transformer and reinforcement learning approaches against Integer Programming on a combinatorial optimisation problem, built reliability-testing pipelines, and ran large experiment sweeps on Apocrita HPC. This work resulted in a first-author preprint (arXiv:2512.11469). In parallel, my undergraduate thesis on the HP Protein Folding problem applies constraint programming and solver-based methods (OR-Tools CP-SAT) to compare classical optimisation with learning-based approaches.

In industry settings, I have delivered production systems with measurable impact: at TinyMagiq, I reduced query latency from 30 seconds to 0.5 seconds and lowered inference cost by over 95\%; at Humanaize, I helped design and deploy resilient, multilingual AI services on AWS using CI/CD, containerisation, and real-time infrastructure. These experiences strengthened my software engineering discipline, experimentation mindset, and ability to collaborate effectively in technical teams.

I am particularly interested in internships across Quantum Software Development, Quantum Algorithms, QML/QNLP, and AI-driven workflows that support quantum applications. I would value the opportunity to contribute my optimisation and ML engineering background while learning from Quantinuum's scientists and engineers at the frontier of quantum computing.

Thank you for your time and consideration. I would welcome the opportunity to discuss my application.

Sincerely,\\[0.4cm]
Pranav Ramanathan

\end{document}
